\documentclass{article}

\def\tightlist{}

\usepackage{booktabs}
\usepackage{caption}	 % http://mirror.easyname.at/ctan/macros/latex/contrib/caption/caption-eng.pdf
\usepackage{graphicx}
\usepackage[T1]{fontenc}
\usepackage{mathptmx}
\usepackage{titlesec}

\newcommand{\sectionbreak}{\clearpage}

%Includes "References" in the table of contents
\usepackage[nottoc]{tocbibind}
\setcounter{section}{1}

%Import the natbib package and sets a bibliography style
\usepackage[square,numbers]{natbib}
\bibliographystyle{abbrvnat}

\begin{document}

\begin{titlepage}
\begin{center}
\vspace*{1cm}

\Huge
\textbf{On the Helpfulness of Using Markdown for Academic Writing}

\vspace{0.8cm}

\Large
\centerline{\textbf{Joe Bloggs}}
\large
\centerline{Lancaster University}
\large
\centerline{Lancaster}
\large
\centerline{England}
\small
\centerline{an.example@lancaster.ac.uk}
\centerline{31-02-2024}

\vspace{0.8cm}

\includegraphics[width=0.4\textwidth]{university}

\end{center}
\end{titlepage}


%% \maketitle

\section*{Abstract}
\thispagestyle{plain}
\begin{center}
    \large
    \vspace{0.9cm}
    With more people using LaTeX for reports in Lancaster's SCC
    department, development of a faster and more intuitive writing
    system is the clear next step.
\end{center}
\newpage
\tableofcontents

\section{INTRODUCTION}

Lots of great stuff is available using Markdown \citep{markdown} with
Pandoc \citep{pandoc}! See those references? They are automatically
generated behind the scenes, and use easy IDs to reference in your
Markdown file. YAML is used at the header of the Markdown file, to add
some more attributes such as the abstract and author info.

\subsection{What's this? A subsection?}

Look at that! Using easy headers we can create subsections that show up
in the table of contents. We have one subsection,

\subsubsection{And more}

With more text\ldots{}

No more subsections can be created beyond this level.

\section{Cool Markdown Things}

We have easy \textbf{bold} formatting, \emph{italics}, and even
\emph{\textbf{BOTH}}! Yet still readable. What about code?
\texttt{programs} can be in monospace. I am working on having code
blocks available.

Less distractions should make writing more fun and faster. Most of the
tedium of the report formatting is done:

\begin{itemize}
\tightlist
\item
  Title Page
\item
  Abstract Page
\item
  Table of Contents
\item
  Sections on new pages
\item
  References section at the end
\end{itemize}

So now, get writing (and referencing)! See BibLaTeX docs for how to
write a bibliography (most journals have .bib references available to
copy).

\emph{Enjoy!}

	\bibliography{report}

\end{document}
\endinput
%%
%% End of file `sample-authordraft.tex'.
